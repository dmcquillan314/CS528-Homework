\documentclass[12pt,letterpaper]{article}

\newenvironment{proof}{\noindent{\bf Proof:}}{\qed\bigskip}

\newtheorem{theorem}{Theorem}
\newtheorem{corollary}{Corollary}
\newtheorem{lemma}{Lemma} 
\newtheorem{claim}{Claim}
\newtheorem{fact}{Fact}
\newtheorem{definition}{Definition}
\newtheorem{assumption}{Assumption}
\newtheorem{observation}{Observation}
\newtheorem{example}{Example}
\newcommand{\qed}{\rule{7pt}{7pt}}

\newcommand{\assignment}[4]{
\thispagestyle{plain} 
\newpage
\setcounter{page}{1}
\noindent
\begin{center}
\framebox{ \vbox{ \hbox to 6.28in
{\bf CS440: Introduction to Artificial Intelligence \hfill #1}
\vspace{4mm}
\hbox to 6.28in
{\hspace{2.5in}\large\mbox{Problem Set #2}}
\vspace{4mm}
\hbox to 6.28in
{{\it Handed Out: #3 \hfill Due: #4}}
}}
\end{center}
}

\newcommand{\solution}[6]{
\thispagestyle{plain} 
\newpage
\setcounter{page}{1}
\noindent
\begin{center}
\framebox{ \vbox{ \hbox to 6.28in
{\bf #5 : #6 \hfill #4}
\vspace{4mm}
\hbox to 6.28in
{\hspace{2.5in}\large\mbox{Problem Set #3}}
\vspace{4mm}
\hbox to 6.28in
{#1 \hfill {\it Handed In: #2}}
}}
\end{center}
\markright{#1}
}

\newenvironment{algorithm}
{\begin{center}
\begin{tabular}{|l|}
\hline
\begin{minipage}{1in}
\begin{tabbing}
\quad\=\qquad\=\qquad\=\qquad\=\qquad\=\qquad\=\qquad\=\kill}
{\end{tabbing}
\end{minipage} \\
\hline
\end{tabular}
\end{center}}

\def\Comment#1{\textsf{\textsl{$\langle\!\langle$#1\/$\rangle\!\rangle$}}}


\usepackage{algorithm}
\usepackage{listings}
%\usepackage{algpseudocode}
\usepackage{graphicx,amssymb,amsmath}
\usepackage{epstopdf}

\usepackage{pgf}
\usepackage{tikz}
\usetikzlibrary{arrows,automata}
\usepackage[latin1]{inputenc}
\usepackage{color}
\usepackage{listings}
\lstset{ %
language=pascal,                % choose the language of the code
basicstyle=\footnotesize,       % the size of the fonts that are used for the code
numbers=left,                   % where to put the line-numbers
numberstyle=\footnotesize,      % the size of the fonts that are used for the line-numbers
stepnumber=1,                   % the step between two line-numbers. If it is 1 each line will be numbered
numbersep=5pt,                  % how far the line-numbers are from the code
backgroundcolor=\color{white},  % choose the background color. You must add \usepackage{color}
showspaces=false,               % show spaces adding particular underscores
showstringspaces=false,         % underline spaces within strings
showtabs=false,                 % show tabs within strings adding particular underscores
frame=single,           % adds a frame around the code
tabsize=2,          % sets default tabsize to 2 spaces
captionpos=b,           % sets the caption-position to bottom
breaklines=true,        % sets automatic line breaking
breakatwhitespace=false,    % sets if automatic breaks should only happen at whitespace
escapeinside={\%*}{*)}          % if you want to add a comment within your code
}

\sloppy


\oddsidemargin 0in
\evensidemargin 0in
\textwidth 6.5in
\topmargin -0.5in
\textheight 9.0in

\begin{document}

\solution{Dan McQuillan}{\today}{7}{Summer 2014}{CS 528}{Object Oriented Programming and Design}

\pagestyle{myheadings}  % Leave this command alone

\begin{enumerate}
	\item[\bf{Part 0}]
	
		Click the links above and read the original blog post, and the response from Kent Beck. Then, answer the following briefly (2-3 sentences).
		
		\begin{enumerate}
			\item What are the major complaints that David brings against TDD?
			
			%answer
			That TDD was hurting the architecture/design of his applications.  TDD will lead to overly complex applications and unnecessary abstraction. He also states that we should be focusing more on the design of an application rather than the structure of our unit tests.
			
			\item What does he propose as being the solution?
			
			%answer
			His proposed solution is that we should put less emphasis on unit tests and more emphasis on system tests.  He proposes that we should move away from test first and move more in the direction of testing the responses after the fact.
			
			\item Pick two of the things Kent Beck would miss without using TDD. Can you accomplish those things without the use of TDD?
			
				\begin{enumerate}
					\item Logic errors
					
					This could be accomplished through rigorous testing of the service, but it would be easier to plan out these utilities via unit testing and the test first pattern.
					
					\item Anxiety
					
					This can be accomplished also through rigorous e2e tests, which can be very time consuming.
					
				\end{enumerate}
		\end{enumerate}	
		
	\item[\bf{Part 1}]
	
		\begin{enumerate}
			
			\item What is a Mock Object? Give an example of the use for a mock.
				
				% answer
				A mock object can be used to simulate an example instantiation of an object/class.  An example usage could be when interacting with a ReST service. For example if you have a user authentication service and you would like to test responses from the service to make sure that the correct flow is taken for a response from the service.

			\item At one point, Kent says the following:
				
				Even if I don't know how to implement something I can almost always figure out how to write a test for it. And if I can't figure out how to write a test, I have no business writing it in the first place.

				Do you agree with this sentiment? Why or why not?
				
				% answer
				I don't fully agree with this sentiment.  However, I do agree with Kent's opinion of you should have no business writing it yet.  I say yet because I believe it's more of an issue of you not understanding the problem at hand.  Not knowing how to write the unit tests is just an example of a causality of not understanding the problem.

			\item What is the "red/green/refactor" loop?
				
				% answer
				Red is after writing the tests they will failure. Then green is working on it until it works.  Once it's working it may not be in a "perfect" state so you should then refactor to clean up the code to make it more readable/reusable.  It is noted as a loop since this process should be repeated until the desired results have been reached.

			\item Which way of arriving at a test suite is better, in your opinion: "going through the tests" (test-first) or "going to the tests" (test-after)? Does it matter at all? Does this depend on the situation? (If so, in what way/ways?)
				
				% answer
				I think it depends on the situation and the task at hand.  If for example you're writing a service that acts as a utility and you fully know what needs to be implemented and all the use cases.  Then, test first may be the better option.  However, if you're working on a task that is more dependent on a third party, such as an api, where all the use cases are not clear.  Then, a mix of the two might be better.  Test first to create the necessary set up and helpers and test after to do integration and performance testing.

			\item In what cases is TDD not (easily) applicable? Give at least one example. What can you do in these cases?
				
				% answer
				An example where TDD is not easily applicable would be when using live data.  In these cases integration, service, and usability testing can be helpful.  You can also use mock objects in those cases although all the cases may not be caught and tested.

			\item In what ways can an overzealous demand for isolation in unit testing create problems? Is this inherent in unit-testing or TDD?
				
				% answer
				By having a demand for isolation you are making sure that each component works when isolated from the rest of the components.  However you are not testing these components when they are communicating with each other which can cause some edge cases to arise.  This is inherent mainly in unit testing as TDD will allows you to thoroughly analyze the problem before beginning on development.  However, it can also appear in TDD in edge cases that a developer may not expect.

			\item Why should you have an automated unit test suite? Give at least two reasons.
							
				% answer
				As the code base grows on larger projects it helps to have to test suite to make sure changes to the code do not break the core functional aspects of the application.  This can help by integrating the unit tests in the build of the application.  Another reason is because it helps to document the functionality of the component you are unit testing.

		\end{enumerate}
	
	\item[\bf{Part 2}]
	
		\begin{enumerate}
			\item What is an MVC application? Where did this design pattern originate?

				% answer
				A simple explanation of MVC is that you have a Model, View and Controller.  The controller manipulates the model.  The model updates the view.  When the view changes it uses and interacts with the controller.
				
				From wikipedia: \\
				MVC was one of the seminal insights in the early development of graphical user interfaces, and one of the first approaches to describe and implement software constructs in terms of their responsibilities. 
				
			\item Can an application have a worse design because of an attempt to make it more testable? Or is it the case that something that is more testable is always a better design?

				% answer
				I don't believe that an application can have a worse design if it is made more testable.  I think that if it is made more testable(using a proper testing strategy) and it is broken up into lexical modules then the design will not suffer.  However, if it is made such that it is broken up into overly simplified modules it can make the design of an application harder to understand.

			\item How does TDD influence the design of code? Give a positive and a negative example.

				% answer
				It can help a programmer to figure out a logical way to structure and separate your application into modules. A positive example could be when writing a larger application it forces you to think out the design of the application beforehand.  However, a negative example is based on the same reason.  Since you have to think out the design of the application, you will be breaking up your application into unit testable components.  For example, if you are writing a simple Customer class you might have to then write an interface, a mock class, some IoC configuration and then tests for each.

			\item David correlates the size of a codebase with how easy it is to change it. Does this correlation ring true, in your experience?

				% answer
				I think that it depends on the situation.  In some cases easy to understand and change code can be written in a few lines when the problem at hand is simple.  However, for larger applications a smaller size of the codebase can lessen how orthogonal the application is. 

			\item Compare the situation David brings up of "test-induced design damage" coming from a desire for isolation in TDD with the concept of "speculative generality". Are they similar? the same? why/why not?

				% answer
				They are in most cases the same.  Speculative generality can be seen when the only callers of a method are test cases.  Such methods could also be seen as test induced design damage.  Test induced design damage also has the same result since in both cases the problem is over abstracted to allow for testing every aspect of the problem in isolation.

			\item What is cohesion? What is coupling? In what ways can more of one mean less of the other?

				% answer
				Coupling is how much a module relies on another module. Cohesion is how much parts of a module belong together.  Being low coupling would mean that changing something major in one class should not affect the other. High coupling would make your code difficult to make changes as well as to maintain it, as classes are coupled closely together, making a change could mean an entire system revamp.

			\item Kent says,

			Difficulty testing is a symptom of poor design.

			Do you agree?

				% answer
				I agree with his statement since if a module is easy to test that means that it will usually have high cohesion.  This is because the higher the cohesion the larger the isolation of the module.  If a module is harder to test that could be a result of having higher coupling since it may be harder to test due to a need for mock objects to test it.

			\item Kent says he is optimistic in that he believes that there is always some design insight that exists that will result in a design that is simultaneously more easily testable as well as better structured.

			Do you share his optimism?

				% answer
				I share his optimism for the hope that a design insight will eventually let a code base's design not suffer when making a code base more testable.

		\end{enumerate}
	
	\item[\bf{Part 3}]
		
		\begin{enumerate}
			\item Martin identifies three important forms of feedback for software development. What were they, and how well is each one served by having a unit test suite?

				% answer

			\item How has Quality Assurance (QA) for programming changed over time? In particular, what role did TDD and self-testing code play in this change?

				% answer

			\item What does David mean when he refers to ?criticality?? How should the ?criticality? of your software influence your testing?

				% answer

			\item Describe how it's possible for code with 100\% test coverage with all tests passing can still have bugs.

				% answer

		\end{enumerate}
		
	\item[\bf{Part 4}]
		
		\begin{enumerate}
			\item Kent mentions a good measure for understanding how important a given test is for a codebase. What is it? How could you determine its value for a given test? Using this measure, how could you numerically define ?overtesting??

				% answer

			\item Describe Martin?s trick used to identify if a given line of code is tested by the current test suite.

				% answer

			\item How is the ratio of functional code to test code influenced by coupling in the system?

				% answer

			\item What are the symptoms of overtested code? undertested code?

				% answer

			\item How does the ?audience? of a codebase change its testing tradeoffs?

				% answer

			\item Describe the difference between placing unit tests as the ?authority? of the system differs from placing the functional code as the ?authority? of the system. Which places a larger focus on refactoring?

				% answer

			\item What is one situation where you would be more happy to lose the tests and keep the code? What is one situation where you would be more happy to lose the code and keep the tests? What is different about these two situations?

				% answer

			\item David thinks the testing battle has been won, and Martin isn?t so convinced. Do you believe that we?ve ?won? the battle to get people to unit test their code? Why or why not?

				% answer

		\end{enumerate}

\end{enumerate}

\end{document}

